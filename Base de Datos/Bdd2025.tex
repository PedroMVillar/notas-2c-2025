\documentclass{article}
\usepackage[a4paper,margin=1.5cm]{geometry}
\usepackage{fontspec}
\usepackage[most]{tcolorbox}
\usepackage{xcolor}

% - Colores para primer parcial - 
\definecolor{mybg}{HTML}{F4978e}
\definecolor{mytemp}{HTML}{F08080}
\colorlet{mybor}{mytemp!60!black}

\setmainfont{Calibri} % o "Arial", "Times New Roman", "Consolas", etc.

\newcounter{clasenro}
\newcommand{\ClaseBoxAuto}[3][]{%
  \refstepcounter{clasenro}%
  \begin{tcolorbox}[
    enhanced, leftrule=10mm, sharp corners=northwest,
    colframe=mybor, colback=mybg,
    overlay={\node[anchor=center,text=white,font=\bfseries\Large]
      at ([xshift=5mm]frame.west) {\theclasenro};},
    #1
  ]
    \begin{center}
      Clase del \textsc{#2} \,-\, \textbf{#3}
    \end{center}
  \end{tcolorbox}%
}

\definecolor{defFrame}{HTML}{2E7D32} % verde sobrio (marco/franja definiciones)
\definecolor{defBack}{HTML}{EEF6EE}  % verde muy claro (fondo definiciones)

\definecolor{obsFrame}{HTML}{8D6E63} % marrón suave (marco/franja observaciones)
\definecolor{obsBack}{HTML}{F7F2EF}  % crema muy claro (fondo observaciones)

\definecolor{exFrame}{HTML}{546E7A}  % azul grisáceo (marco/franja ejercicios)
\definecolor{exBack}{HTML}{F2F6F8}   % gris-azulado muy claro (fondo ejercicios)

% ==== Estilo base para que todo combine ====
\tcbset{
  mybase/.style={
    enhanced,
    sharp corners=northwest,
    coltitle=black,
    fonttitle=\bfseries,
    boxsep=4pt,   % padding interno
    left=6pt,     % margen interior izquierdo
    right=6pt,
    top=6pt,
    bottom=6pt,
  }
}

% ==== Definición ====
\newtcolorbox{definicion}[1][]{%
  mybase,
  colframe=defFrame,
  colback=defBack,
  borderline west={10mm}{0pt}{defFrame}, % franja izquierda
  title=\textbf{Definición},
  % Sello "DEF" en la franja (discreto)
  overlay={
    \node[anchor=center,text=white,font=\bfseries\footnotesize]
      at ([xshift=5mm]frame.west) {DEF};
  },
  #1
}

% ==== Observación ====
\newtcolorbox{observacion}[1][]{%
  mybase,
  colframe=obsFrame,
  colback=obsBack,
  borderline west={10mm}{0pt}{obsFrame},
  title=\textbf{Observación},
  overlay={
    \node[anchor=center,text=white,font=\bfseries\footnotesize]
      at ([xshift=5mm]frame.west) {OBS};
  },
  #1
}

% ==== Ejercicio teórico ====
\newcounter{exth}
\newtcolorbox{ejercicioTeorico}[2][]{%
  mybase,
  colframe=exFrame,
  colback=exBack,
  borderline west={10mm}{0pt}{exFrame},
  before upper=\refstepcounter{exth},
  title={\textbf{Ejercicio teórico \theexth}\IfValueT{#2}{\,: \,#2}},
  overlay={
    \node[anchor=center,text=white,font=\bfseries\footnotesize]
      at ([xshift=5mm]frame.west) {EX};
  },
  #1
}


\begin{document}
% Title page
\begin{titlepage} 
	\newcommand{\HRule}{\rule{\linewidth}{0.5mm}}
	
	\center 
	
	%------------------------------------------------
	%	Headings
	%------------------------------------------------
	
	\textsc{\LARGE FaMAF}\\[1.5cm] 
	
	\textsc{\Large Notas de Clase}\\[0.5cm] 
	
	\textsc{\large 2025}\\[0.5cm] 
	
	%------------------------------------------------
	%	Title
	%------------------------------------------------
	
	\HRule\\[0.4cm]
	
	{\huge\bfseries Bases de Datos}\\[0.4cm] 
	
	\HRule\\[1.5cm]
	
	%------------------------------------------------
	%	Author(s)
	%------------------------------------------------
	
	\begin{minipage}{0.4\textwidth}
		\begin{flushleft}
			\large
			\textit{Estudiante}\\
			Pedro \textsc{Villar} 
		\end{flushleft}
	\end{minipage}
	
	
	%------------------------------------------------
	%	Date
	%------------------------------------------------
	
	\vfill\vfill\vfill 
	
	{\large\today} 

	
	%----------------------------------------------------------------------------------------
	
	\vfill 
	
\end{titlepage}
% Archivo generado automaticamente NO TOCAR

\ClaseBoxAuto{Miércoles 13 de Agosto}{Introducción a SystemVerilog}

\ClaseBoxAuto{Viernes 15 de Agosto}{Introducción a SystemVerilog}



\end{document}
