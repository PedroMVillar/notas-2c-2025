\newpage
\ClaseBoxAuto{Jueves 14 de Agosto}{Introducción}

Algebra de tabla es mas simple que sql pero mas formal

\section{Lenguajes de Consulta}

\section{Sistema gestor de bases de datos relacionales}

\subsection{Gestor de almacenamiento}

\subsection{Niveles lógico y físico}

\subsubsection{Nivel lógico y físico}

\paragraph{Nivel físico}
Completar

\paragraph{Nivel lógico}
Implementa el \textit{cómo acceder} a los datos, trabaja con formatos por bloques y registros, y con la estructura física del almacenamiento. El acceso eficiente a los datos es implementado mediante \textbf{índices}.

El índice si es grande, puede cargarse una parte en RAM y otra parte en disco, en cambio si es pequeño se puede cargar todo en RAM. Normalmente las tablas como índices están en archivos físicos, y el dónde se encuentra depende del sistema operativo y del sistema de archivos.

\begin{tcolorbox}[colback=white,colframe=red,sharp corners]

Arriba del nivel lógico tenemos la aplicación de base de datos, que es la que interactúa con el usuario, y debajo del nivel físico tenemos el sistema operativo (manejo de memoria, archivos, etc). El gestor de almacenamiento se encuentra en medio, provee una interfaz entre el almacenamiento físico y las consultas y actualizaciones.

Va a \textbf{manejar} tanto el almacenamiento en disco como en memoria.

Las responsabilidades del gestor de almacenamiento son:
\begin{itemize}
    \item Dar forma adecuada de \textbf{almacenar los datos}. Organizar datos en archivos.
    \item Definir \textbf{cómo acceder al almacenamiento} de manera eficiente.
    \item \textbf{Mantener índices, archivos y catálogos}.
    \item \textbf{Optimizar el acceso} a los datos. Mediante el uso de índices, por ejemplo operaciones para crear índices, indexar datos, etc.
    \item \textbf{Administrar el espacio} en memoria.
\end{itemize}

Y las funciones específicas son:
\begin{itemize}
    \item \textbf{Manejo de archivos e índices}.
    \item \textbf{Manejar catálogos}, es decir metadatos.
    \item \textbf{Admimistración de memoria}, mediante \textit{pool de bufferes}, \textit{caching} y \textit{paginación}.
\end{itemize}

\end{tcolorbox}

\subsection{Procesamiento de consultas}

La idea principal es consultar el lenguaje declarativo como SQL, y esta va a ser transformada en una ejecución eficiente de los datos. El algoritmo va a \textbf{acceder} a funciones del \textit{gestor de almacenamiento}.

Ahora bien, hay que dividir este procesamiento en varias etapas y por cada una de estas dar componentes que se encarguen de realizar las tareas específicas. Estas etapas son:

\begin{enumerate}
    \item \textbf{Parseo y traducción}:
    \item \textbf{Optimización de consultas}:
    \item \textbf{Evaluación}:
\end{enumerate}

\begin{tcolorbox}[colback=white,colframe=red,sharp corners]
En resumen, el procesamiento de consultas hace referencia a todo lo implicado en la extracción de datos de una base de datos. Incluyendo la \textbf{traducción} de consultas expresadas en lenguajes de bases de datos de alto nivel en expresiones implementadas en el nivel físico del sistema, la \textbf{optimización} de estas expresiones para mejorar la eficiencia de la ejecución, y la \textbf{evaluación} de las expresiones optimizadas para recuperar los datos solicitados.
\end{tcolorbox}

\subsection{Gestión de transacciones}

